\documentclass[a4paper,12pt,preview]{report} %размер бумаги устанавливаем А4, шрифт 12пунктов
\usepackage[english,russian]{babel}%используем русский и английский языки с переносами 	
\usepackage[T2A]{fontenc}
\usepackage{changepage}
\usepackage{lipsum}
\usepackage{indentfirst}
\usepackage[labelsep=period]{caption}
\usepackage{amsmath}
\usepackage{textcomp}
%\usepackage{enumitem}
\usepackage[utf8]{inputenc}%включаем свою кодировку: koi8-r или utf8 в UNIX, cp1251 в Windows
\usepackage[english,russian]{babel}%используем русский и английский языки с переносами 	
\usepackage{amssymb,amsfonts,amsmath,mathtext,cite,enumerate,float} %подключаем нужные пакеты расширений
\usepackage{graphicx} %хотим вставлять в диплом рисунки?
\usepackage{ragged2e}
\usepackage{indentfirst}
\usepackage{titlesec}
\graphicspath{{images/}}%п\usepackage{trd}уть к рисункам

\makeatletter
\renewcommand{\@biblabel}[1]{#1.} % Заменяем библиографию с квадратных скобок на точку:
\makeatother

\usepackage{geometry} % Меняем поля страницы
\geometry{left=2cm}% левое поле
\geometry{right=1.5cm}% правое поле
\geometry{top=1cm}% верхнее поле
\geometry{bottom=2cm}% нижнее поле

\renewcommand{\theenumi}{\arabic{enumi}}% Меняем везде перечисления на цифра.цифра
\renewcommand{\labelenumi}{\arabic{enumi}}% Меняем везде перечисления на цифра.цифра
\renewcommand{\theenumii}{.\arabic{enumii}}% Меняем везде перечисления на цифра.цифра
\renewcommand{\labelenumii}{\arabic{enumi}.\arabic{enumii}.}% Меняем везде перечисления на цифра.цифра
\renewcommand{\theenumiii}{.\arabic{enumiii}}% Меняем везде перечисления на цифра.цифра
\renewcommand{\labelenumiii}{\arabic{enumi}.\arabic{enumii}.\arabic{enumiii}.}% Меняем везде перечисления на цифра.цифра

\newcommand{\doublerule}[1][.4pt]{%
	\noindent
	\makebox[0pt][l]{\rule[.7ex]{\linewidth}{#1}}%
	\rule[.3ex]{\linewidth}{#1}}



\titleformat{\chapter}[hang]
{\normalfont\huge\bfseries}{\thechapter.}{20pt}{}

\renewcommand*\thesection{\arabic{section}}

\newenvironment{boenumerate}
{\begin{enumerate}\renewcommand\labelenumi{\textbf\theenumi}}
	{\end{enumerate}}


\begin{document}
	
	\begin{center}
		Министерство образования и науки РФ \\
		Федеральное государственное автономное образовательное учреждение высшего профессионального образования <<НИТУ МИСиС>>\\
		Институт ИТАСУ\\
		Кафедра Инженерной кибернетики\\
	\end{center}
	
	
	\vfill
	
	\begin{center}
		\Large\textbf{Отчет \textnumero 2 \\
			по курсу <<Программная инженерия>>\\
			тема "Генератор фракталов"}
	\end{center}
	
	\vfill
	
	\begin{FlushRight}
		Выполнил\\
		Студент группы \\
		БПМ-16-2 \\
		Фадеев А.Ю. \\
		[\baselineskip]
		Проверил: \\
		Широков А.И. \\
		[9\baselineskip]
	\end{FlushRight}
	
	
	\begin{center}
		Москва 2020
	\end{center}
	
	\thispagestyle{empty}
	\newpage
	
	%\tableofcontents
	%\newpage
	
	Разработка технического задания на создание программ.
	
	Цель работы: ознакомиться с правилами написания технического задания.
	
	
	\begin{center}
		\textbf{Техническое задание на разработку системы <<Генератор фракталов>>.}
	\end{center}
	
	\begin{boenumerate}
		\item \textbf{Введение.} 
		
		Работа выполняется в рамках учебной дисциплины «Программная инженерия» кафедры Инженерной кибернетики.
		
		\item \textbf{Основание для разработки.}
		
		Данный раздел содержит описание причин, побудивших к разработке проектируемого программного обеспечения. 
		
		\begin{enumerate}
			\item Основанием для данной работы служат требования учебной дисциплины «Программная инженерия».
			\item Наименование работы: «Генератор фракталов».
			\item Исполнители: Фадеев Александр, студент НИТУ «МИСиС», институт ИТАСУ, группа БПМ-16-2
			\item Соисполнители: нет.
		\end{enumerate}
	
		\item \textbf{Назначение разработки.}
		
		Создание программной системы генератора фракталов, удовлетворяющей требованиям к поставленной задаче.
		
		\item \textbf{Технические требования.}
		
		\begin{enumerate}
			\item Требования к функциональным характеристикам.
			
			На основании ГОСТ 19.102-77 приведем список требований к функциональным характеристикам.
			
			\begin{enumerate}
				\item Состав выполняемых функций.
				
				Система должна поддерживать выполнение следующих вариантов использования:
				
				\begin{itemize}
					\item Задания исходного множества и набора желаемых преобразований подобия.
					\item Визуализации полученного фрактала.
					\item Сохранения изображения фрактала в файл.
					\item Изменения масштаба изображения.
					\item Отрисовки координатной сетки.
					\item Изменения цвета изображения.
					\item Изменение количества итераций применения преобразований подобия.
				\end{itemize}
			
				\item Организация входных и выходных данных.
				
				Входными данными является исходный полигон и все преобразования подобия. Они содержатся в файле "settings.txt" в соответствии со следующим форматов:
				
				\begin{itemize}
					\item В первой строке написано одно число n – кол-во точек начального многоугольника.
					\item В следующих n строчках задаются точки полигона в порядке обхода по или против часовой стрелки.
					\item В следующей строке написано одно число $m$ – кол-во преобразований.
					\item В следующих $m$ строках заданы матрицы преобразования $A_i$ и $b_i$ (преобразование $f_i = A_i \times x + b_i$). 
				\end{itemize} 
			
				
				Выходными данными является полученное изображение.
				
			
			\end{enumerate}
		
		\item Требования к надежности.
		
		Необходима проверка корректности вносимых пользователем изменений; в случае возникновения ошибки требуется визуализировать соответствующее сообщение с указанием возможной причины ошибки.
		
		\item Условия эксплуатации и требования к составу и параметрам технических средств.
			
		В силу того, что система предназначена для использования в рамках учебного процесса, необходимо использование квалифицированным специалистом, обладающим достоверными сведениями, касающимися работы с системой.
		
		Требования к составу и параметрам технических средств уточняются на этапе эскизного проектирования системы.
		
		\item Требования к информационной и программной совместимости.
		
		Программа должна работать на платформах Windows 7/8/10.
		
		\item Требования к транспортировке и хранению.
		
		Программа поставляется на лазерном носителе информации. Программная документация поставляется в электронном и печатном виде.
		
		\item Специальные требования.
		
		Программное обеспечение должно иметь дружественный интерфейс, рассчитанный на пользователя средней квалификации (по уровню компьютерной грамотности).
		
		Ввиду объемности проекта задачи предполагается решать поэтапно, при этом модули ПО, созданные в разное время, должны предполагать возможность наращивания системы и быть совместимы друг с другом, поэтому документация на принятое эксплуатационное ПО должна содержать полную информацию, необходимую для работы программистов с ним.
		
		Язык программирования – по выбору исполнителя, должен обеспечивать возможность интеграции программного обеспечения с некоторыми видами периферийного оборудования.
		
		\end{enumerate}
		
	\item \textbf{Требования к программной документации.}
	
	Основными документами, регламентирующими разработку будущих программ, должны быть документы Единой Системы Программной документации (ЕСПД); руководство пользователя, руководство администратора; описание применения.
	
	\item \textbf{Технико-экономические показатели.}
	
	Эффективность системы определяется удобством использования системы для отрисовки фракталов.
	
	\item \textbf{Порядок контроля и приемки.}
	
	Перед передачей готовой системы заказчику, необходимо провести тестирование. В случае нахождения заказчиком проблем, последний имеет право заявить отказ от принятия с четким указанием причины. В случае обоснованного отказа исполнитель обязуется доработать систему.
	
	
	
	\end{boenumerate}
	
	
\end{document}


