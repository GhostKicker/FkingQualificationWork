\documentclass[a4paper,12pt,preview]{report} %размер бумаги устанавливаем А4, шрифт 12пунктов
\usepackage[english,russian]{babel}%используем русский и английский языки с переносами 	
\usepackage[T2A]{fontenc}
\usepackage{changepage}
\usepackage{lipsum}
\usepackage{indentfirst}
\usepackage[labelsep=period]{caption}
\usepackage{amsmath}
\usepackage{textcomp}
\usepackage[utf8]{inputenc}%включаем свою кодировку: koi8-r или utf8 в UNIX, cp1251 в Windows
\usepackage[english,russian]{babel}%используем русский и английский языки с переносами 	
\usepackage{amssymb,amsfonts,amsmath,mathtext,cite,enumerate,float} %подключаем нужные пакеты расширений
\usepackage{graphicx} %хотим вставлять в диплом рисунки?
\usepackage{ragged2e}
\usepackage{indentfirst}
\usepackage{titlesec}
\graphicspath{{images/}}%п\usepackage{trd}уть к рисункам

\makeatletter
\renewcommand{\@biblabel}[1]{#1.} % Заменяем библиографию с квадратных скобок на точку:
\makeatother

\usepackage{geometry} % Меняем поля страницы
\geometry{left=2cm}% левое поле
\geometry{right=1.5cm}% правое поле
\geometry{top=1cm}% верхнее поле
\geometry{bottom=2cm}% нижнее поле

\renewcommand{\theenumi}{\arabic{enumi}}% Меняем везде перечисления на цифра.цифра
\renewcommand{\labelenumi}{\arabic{enumi}}% Меняем везде перечисления на цифра.цифра
\renewcommand{\theenumii}{.\arabic{enumii}}% Меняем везде перечисления на цифра.цифра
\renewcommand{\labelenumii}{\arabic{enumi}.\arabic{enumii}.}% Меняем везде перечисления на цифра.цифра
\renewcommand{\theenumiii}{.\arabic{enumiii}}% Меняем везде перечисления на цифра.цифра
\renewcommand{\labelenumiii}{\arabic{enumi}.\arabic{enumii}.\arabic{enumiii}.}% Меняем везде перечисления на цифра.цифра

\newcommand{\doublerule}[1][.4pt]{%
	\noindent
	\makebox[0pt][l]{\rule[.7ex]{\linewidth}{#1}}%
	\rule[.3ex]{\linewidth}{#1}}



\titleformat{\chapter}[hang]
{\normalfont\huge\bfseries}{\thechapter.}{20pt}{}

\renewcommand*\thesection{\arabic{section}}

\begin{document}
	
	\begin{center}
		Министерство образования и науки РФ \\
		Федеральное государственное автономное образовательное учреждение высшего профессионального образования <<НИТУ МИСиС>>\\
		Институт ИТАСУ\\
		Кафедра Инженерной кибернетики\\
	\end{center}
	
	
	\vfill
	
	\begin{center}
		\Large\textbf{Отчет \textnumero 1 \\
			по курсу <<Программная инженерия>>\\
		тема "Генератор фракталов"}
	\end{center}
	
	\vfill
	
	\begin{FlushRight}
		Выполнил\\
		Студент группы \\
		БПМ-16-2 \\
		Фадеев А.Ю. \\
		[\baselineskip]
		Проверил: \\
		Широков А.И. \\
		[9\baselineskip]
	\end{FlushRight}
	
	
	\begin{center}
		Москва 2020
	\end{center}
	
	\thispagestyle{empty}
	\newpage
	
	\tableofcontents
	\newpage
	
	
	\section{Цель}
	
	Создать систему, позволяющую визуализировать заданные пользователем с помощью набора преобразований подобия и исходного множества фракталы.
	
	
	\section{Формулировка задания}
	
	Система должна поддерживать выполнение следующих вариантов использования:
	
	\begin{itemize}
		\item Задания исходного множества и набора желаемых преобразований подобия.
		\item Визуализации полученного фрактала.
		\item Сохранения изображения фрактала в файл.
		\item Изменения масштаба изображения.
		\item Отрисовки координатной сетки.
		\item Изменения цвета изображения.
		\item Изменение количества итераций применения преобразований подобия.
	\end{itemize}
	
	\section{Взаимодействие с пользователем}
	
	Программа должна предоставлять возможность взаимодействия с конечным пользователем посредством графического интерфейса. Интерфейс должен быть наглядным, понятным, предоставлять доступ ко всем функциям, которые предусмотрены в разделе «Формулировка задания».
	
	\section{Описание предполагаемого результата работы}
	
	Реализация представляет собой программу, предусматривающую взаимодействие всех основных элементов между собой и реализацию всех необходимых функций.
	
	\section{Средства разработки}
	
	\begin{enumerate}
		\item Visual Studio 2019 
		
		\item Компилятор C++ 17 MVSC
		
		\item OpenCV — библиотека алгоритмов компьютерного зрения, обработки изображений и численных алгоритмов общего назначения с открытым кодом
		
	\end{enumerate}
	
	\section{Интерфейс}
	
	Взаимодействие с пользователем осуществляется посредством графического интерфейса, представляющего собой окно визуализации и два дополнительных окна настроек, предоставляющих возможность посредством передвижения различных "ползунков" изменять те или иные характеристики основного изображения. Список регулируемых характеристик:
	
	\begin{enumerate}
		\item Окно основных настроек:
		
		\begin{itemize}
			\item Масштаб
			\item Число итераций применения преобразований подобия
			\item Сдвиг центра координат по оси Х
			\item Сдвиг центра координат по оси Y
			\item Ширина изображения
			\item Высота изображения
			\item Режимы координатной сетки
			\item Масштаб координатной сетки
		\end{itemize}
		
		\item Окно дополнительных настроек:
		
		\begin{itemize}
			\item Цвет фона изображения
			\item Цвет фрактала
		\end{itemize}
		
	\end{enumerate}
	
	
	\section{Входные данные}
	Исходный многоугольник, а также все преобразования подобия описываются в файле \textit{"settings.txt"}.
	
	
	\section{Краткое описание алгоритма}
	К заданному в \textit{"settings.txt"} полигону применяются преобразования подобия (являющиеся аффинными) заданное число итераций раз, полученное изображение визуализируется в основном окне.  
	
	
	
	
\end{document}


