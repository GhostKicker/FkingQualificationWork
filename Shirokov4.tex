\documentclass[a4paper,12pt,preview]{report} %размер бумаги устанавливаем А4, шрифт 12пунктов
\usepackage[english,russian]{babel}%используем русский и английский языки с переносами 	
\usepackage[T2A]{fontenc}
\usepackage{changepage}
\usepackage{lipsum}
\usepackage{indentfirst}
\usepackage[labelsep=period]{caption}
\usepackage{amsmath}
\usepackage{textcomp}
%\usepackage{enumitem}
\usepackage[utf8]{inputenc}%включаем свою кодировку: koi8-r или utf8 в UNIX, cp1251 в Windows
\usepackage[english,russian]{babel}%используем русский и английский языки с переносами 	
\usepackage{amssymb,amsfonts,amsmath,mathtext,cite,enumerate,float} %подключаем нужные пакеты расширений
\usepackage{graphicx} %хотим вставлять в диплом рисунки?
\usepackage{ragged2e}
\usepackage{indentfirst}
\usepackage{titlesec}
\graphicspath{{images/}}%п\usepackage{trd}уть к рисункам

\makeatletter
\renewcommand{\@biblabel}[1]{#1.} % Заменяем библиографию с квадратных скобок на точку:
\makeatother

\usepackage{geometry} % Меняем поля страницы
\geometry{left=2cm}% левое поле
\geometry{right=1.5cm}% правое поле
\geometry{top=1cm}% верхнее поле
\geometry{bottom=2cm}% нижнее поле

\renewcommand{\theenumi}{\arabic{enumi}}% Меняем везде перечисления на цифра.цифра
\renewcommand{\labelenumi}{\arabic{enumi}}% Меняем везде перечисления на цифра.цифра
\renewcommand{\theenumii}{.\arabic{enumii}}% Меняем везде перечисления на цифра.цифра
\renewcommand{\labelenumii}{\arabic{enumi}.\arabic{enumii}.}% Меняем везде перечисления на цифра.цифра
\renewcommand{\theenumiii}{.\arabic{enumiii}}% Меняем везде перечисления на цифра.цифра
\renewcommand{\labelenumiii}{\arabic{enumi}.\arabic{enumii}.\arabic{enumiii}.}% Меняем везде перечисления на цифра.цифра

\newcommand{\doublerule}[1][.4pt]{%
	\noindent
	\makebox[0pt][l]{\rule[.7ex]{\linewidth}{#1}}%
	\rule[.3ex]{\linewidth}{#1}}



\titleformat{\chapter}[hang]
{\normalfont\huge\bfseries}{\thechapter.}{20pt}{}

\renewcommand*\thesection{\arabic{section}}

\newenvironment{boenumerate}
{\begin{enumerate}\renewcommand\labelenumi{\textbf\theenumi}}
	{\end{enumerate}}


\begin{document}
	
	\begin{center}
		Министерство образования и науки РФ \\
		Федеральное государственное автономное образовательное учреждение высшего профессионального образования <<НИТУ МИСиС>>\\
		Институт ИТАСУ\\
		Кафедра Инженерной кибернетики\\
	\end{center}
	
	
	\vfill
	
	\begin{center}
		\Large\textbf{Отчет \textnumero 4 (Технический проект) \\
			по курсу <<Программная инженерия>>\\
			тема "Генератор фракталов"}
	\end{center}
	
	\vfill
	
	\begin{FlushRight}
		Выполнил\\
		Студент группы \\
		БПМ-16-2 \\
		Фадеев А.Ю. \\
		[\baselineskip]
		Проверил: \\
		Широков А.И. \\
		[9\baselineskip]
	\end{FlushRight}
	
	
	\begin{center}
		Москва 2020
	\end{center}
	
	\thispagestyle{empty}
	\newpage
	
	\tableofcontents
	\newpage
	
	\section{Общие положения.}
	
	Ниже приведена основная информация о проектируемом ПО.
	
	\subsection{Наименование системы.}
	Наименование системы - "Генератор фракталов".
	
	\subsection{Основание для проведения работ.}
	Основанием для данной работы служат требования учебной дисциплины "Программная инженерия".
	
	\subsection{Наименование организации -- заказчика и разработчика.}
	\subsubsection{Заказчик}
	НИТУ «МИСиС», институт ИТАСУ, кафедра Инженерной кибернетики (доцент Широков А. И.).
	\subsubsection{Разработчик}
	Исполнитель: Фадеев Александр, студент НИТУ «МИСиС», институт ИТАСУ, группа БПМ-16-2.
	
	\subsection{Цели, назначение и область использования системы.}
	Создание программной системы расписания занятий, удовлетворяющей требованиям к поставленной задаче.
		
	\subsection{Нормативные ссылки.}
	При техническом проектировании использовались следующие нормативно-технические документы:
	
	\begin{itemize}
		\item Техническое задание.
		\item Пояснительная записка к эскизному проекту.
		\item ГОСТ 19.102-77 (ЕСПД).
	\end{itemize}


	\section{Основные технические решения.}
		
	Готовое продукт должен подчиняться следующим алгоритмическим решениям и иметь следующие элементы интерфейса:
	
	\subsection{Разработка алгоритма решения задачи.}
		
		К заданному в \textit{"settings.txt"} полигону применяются преобразования подобия (являющиеся аффинными) заданное число итераций раз, полученное изображение визуализируется в основном окне.
		
	\subsection{Интерфейс.}
		
		Представляет собой окно визуализации и два дополнительных окна настроек, предоставляющих возможность посредством передвижения различных "ползунков" изменять те или иные характеристики основного изображения. Список регулируемых характеристик:
		
		Окно основных настроек:
		
		\begin{itemize}
			\item Масштаб.
			\item Число итераций применения преобразований подобия.
			\item Сдвиг центра координат по оси Х.
			\item Сдвиг центра координат по оси Y.
			\item Ширина изображения.
			\item Высота изображения.
			\item Режимы координатной сетки.
			\item Масштаб координатной сетки.
		\end{itemize}
		
		Окно дополнительных настроек:
		
		\begin{itemize}
			\item Цвет фона изображения.
			\item Цвет фрактала.
		\end{itemize}
		
		
	
	\section{Решение по режимам функционирования, работы системы.}
	
	Система будет функционировать в однопользовательском режиме, а также будет способна:
	
	\begin{itemize}
		\item Задания исходного множества и набора желаемых преобразований подобия.
		\item Визуализировать заданные фракталы.
		\item Сохранения изображения фрактала в файл.
	\end{itemize}
	
	\section{Решения по численности, квалификации и функциям персонала АС.}
	
	Указанные решения должны удовлетворять требованиям, приведенным в техническом задании на разработку системы.
	
	\section{Состав функций комплексов задач, реализуемых системой.}
	
	
	\begin{itemize}
		\item Задания исходного множества и набора желаемых преобразований подобия.
		\item Визуализации полученного фрактала.
		\item Сохранения изображения фрактала в файл.
		\item Изменения масштаба изображения.
		\item Отрисовки координатной сетки.
		\item Изменения цвета изображения.
		\item Изменение количества итераций применения преобразований подобия.
	\end{itemize}
	
	\section{Решения по составу программных средств, языкам деятельности, алгоритмам процедур и операций и методам их реализации.}
	
	Для реализации будут использованы следующие средства разработки:
	
	\subsection{Средства разработки.}
		
		\begin{itemize}
			\item Visual Studio 2019.
			
			\item Компилятор C++ 17 MVSC.
			
			\item OpenCV — библиотека алгоритмов компьютерного зрения, обработки изображений и численных алгоритмов общего назначения с открытым кодом.
			
		\end{itemize}
		
	\subsection{Пользовательский интерфейс.}
		
		Пользовательский интерфейс будет реализован с помощью встроенных средств Open-Source библиотеки OpenCV.
		
		
		\begin{itemize}
			\item Задания исходного множества и набора желаемых преобразований подобия.
			
			Предусматривается текстовый файл, в который пользователь задает данные. \\
			
			\item Визуализации полученного фрактала.
			
			Окно с визуализацией генерируется при запуске программы. \\
			
			\item Сохранения изображения фрактала в файл.
			
			Предусматривается наличие кнопки, осуществляющей сохранение полученного изображение в директорию с программой. \\
			
			\item Изменения масштаба изображения.
			
			Предусматривается наличие ползунка, осуществляющее уменьшение или увеличение размера изображения. \\
			
			\item Отрисовки координатной сетки.
			
			Предусматривается наличие ползунка, осуществляющее выбор между различными видами сетки, а именно: отсутствие сетки, только координатные оси, полноценная сетка. \\
			
			\item Изменения цвета изображения.
			
			Предусматривается наличие ползунков, служащих для изменение красного, синего и зеленого каналов фрактала и фона. \\
			
			\item Изменение количества итераций применения преобразований подобия.
			
			Предусматривается наличие ползунка, служащего для регулирования кол-ва итераций применений преобразований.
			
			
		\end{itemize}
		
		
	



	
\end{document}


