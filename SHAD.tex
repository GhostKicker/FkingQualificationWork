\documentclass[a4paper,12pt,preview]{report} %размер бумаги устанавливаем А4, шрифт 12пунктов
\usepackage[english,russian]{babel}%используем русский и английский языки с переносами 	
\usepackage[T2A]{fontenc}
\usepackage{changepage}
\usepackage{lipsum}
\usepackage{indentfirst}
\usepackage[labelsep=period]{caption}
\usepackage{amsmath}
\usepackage{textcomp}
\usepackage[utf8]{inputenc}%включаем свою кодировку: koi8-r или utf8 в UNIX, cp1251 в Windows
\usepackage[english,russian]{babel}%используем русский и английский языки с переносами 	
\usepackage{amssymb,amsfonts,amsmath,mathtext,cite,enumerate,float} %подключаем нужные пакеты расширений
\usepackage{graphicx} %хотим вставлять в диплом рисунки?
\usepackage{ragged2e}
\usepackage{fancyhdr}
\usepackage{indentfirst}
\usepackage{titlesec}
\usepackage{listings}
\lstset{
	basicstyle=\small\ttfamily,
	columns=flexible,
	breaklines=true
}
\graphicspath{{images/}}%п\usepackage{trd}уть к рисункам

\makeatletter
\renewcommand{\@biblabel}[1]{#1.} % Заменяем библиографию с квадратных скобок на точку:
\makeatother

\usepackage{geometry} % Меняем поля страницы
\geometry{left=2cm}% левое поле
\geometry{right=1.5cm}% правое поле
\geometry{top=3cm}% верхнее поле
\geometry{bottom=2cm}% нижнее поле
\usepackage{hyperref}
\hypersetup{
	colorlinks,
	citecolor=black,
	filecolor=black,
	linkcolor=black,
	urlcolor=black
}

\renewcommand{\theenumi}{\arabic{enumi}}% Меняем везде перечисления на цифра.цифра
\renewcommand{\labelenumi}{\arabic{enumi}}% Меняем везде перечисления на цифра.цифра
\renewcommand{\theenumii}{.\arabic{enumii}}% Меняем везде перечисления на цифра.цифра
\renewcommand{\labelenumii}{\arabic{enumi}.\arabic{enumii}.}% Меняем везде перечисления на цифра.цифра
\renewcommand{\theenumiii}{.\arabic{enumiii}}% Меняем везде перечисления на цифра.цифра
\renewcommand{\labelenumiii}{\arabic{enumi}.\arabic{enumii}.\arabic{enumiii}.}% Меняем везде перечисления на цифра.цифра

\newcommand{\doublerule}[1][.4pt]{%
	\noindent
	\makebox[0pt][l]{\rule[.7ex]{\linewidth}{#1}}%
	\rule[.3ex]{\linewidth}{#1}}

\pagestyle{fancy}
\fancyhf{}
\rhead{Фадеев Александр Юрьевич}
\rfoot{\thepage}

\titleformat{\chapter}[hang]
{\normalfont\huge\bfseries}{\thechapter.}{20pt}{}


\begin{document}
	\begin{center}
		\Huge ЗАДАЧА C2
	\end{center}
	
	Задачу можно решить <<динамическим программированием>>.\\
	
	Пусть \texttt{dp[a][b] = prob} - текущее состояние.
	
	\texttt{a} - кол-во выпавших орлов.
	
	\texttt{b} - кол-во выпавших орлов.
	
	\texttt{prob} - вероятность попасть в это состояние. \\
	
	База динамики: \texttt{dp[0][0] = 1}
	
	Переходы динамики:
	
	\texttt{dp[a+1][b] += dp[a][b] * p}
	
	
	\texttt{dp[a][b+1] += dp[a][b] * p} \\
	
	Если у состояния \texttt{a >= 2} и \texttt{b >= 1}, то к ответу прибавляется \texttt{(a + b) * prob } этого состояния. \\
	
	Так как требуется точность 4 зн. после запятой, можно считать результат только для довольно небольших (a + b). \\
	
	Код на C++
	
	\begin{lstlisting}
	
#include <bits/stdc++.h>
using namespace std;

double p = 1.0 / 6;
double q = 5.0 / 6;

int main(){

    double res = 0;
    map<pair<int,int>, double> dp;
    dp[pair<int,int>(0,0)] = 1;
    
    for (int i = 0; i < 10000; ++i){
        if (i % 500 == 0) cerr << i << " " << res << endl;
        map<pair<int,int>, double> dpswap;
        for (auto& it : dp){
            pair<int,int> nxt;
            double next_p;

            nxt = it.first;
            nxt.first += 1;
            next_p = it.second * p;
            if (nxt.first >= 2 && nxt.second >= 1) 
                res += (nxt.first + nxt.second) * next_p;
            else
                dpswap[nxt] += next_p;

            nxt = it.first;
            nxt.second += 1;
            next_p = it.second * q;
            if (nxt.first >= 2 && nxt.second >= 1)
                res += (nxt.first + nxt.second) * next_p;
            else
                dpswap[nxt] += next_p;
        }
        swap(dp, dpswap);
    }

    cout << res;

    return 0;
}

	
	\end{lstlisting}
	


\end{document}


