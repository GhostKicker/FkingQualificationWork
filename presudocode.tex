\documentclass[a4paper,12pt,preview]{report} %размер бумаги устанавливаем А4, шрифт 12пунктов
\usepackage[english,russian]{babel}%используем русский и английский языки с переносами 	
\usepackage[T2A]{fontenc}
\usepackage{changepage}
\usepackage{amssymb}
\usepackage{lipsum}
\usepackage{indentfirst}
\usepackage[labelsep=period]{caption}
\usepackage{amsmath}
\usepackage[utf8]{inputenc}%включаем свою кодировку: koi8-r или utf8 в UNIX, cp1251 в Windows
\usepackage[english,russian]{babel}%используем русский и английский языки с переносами 	
\usepackage{amssymb,amsfonts,amsmath,mathtext,cite,enumerate,float} %подключаем нужные пакеты расширений
\usepackage{graphicx} %хотим вставлять в диплом рисунки?
\usepackage{indentfirst}
\usepackage{titlesec}
\usepackage{chngcntr}
\usepackage{float}
\graphicspath{{images/}}%п\usepackage{trd}уть к рисункам
\usepackage{hyperref}
\hypersetup{
	colorlinks,
	citecolor=black,
	filecolor=black,
	linkcolor=black,
	urlcolor=black
}

\makeatletter
\renewcommand{\@biblabel}[1]{#1.} % Заменяем библиографию с квадратных скобок на точку:
\makeatother

\usepackage{geometry} % Меняем поля страницы
\geometry{left=2cm}% левое поле
\geometry{right=1.5cm}% правое поле
\geometry{top=1cm}% верхнее поле
\geometry{bottom=2cm}% нижнее поле

\renewcommand{\theenumi}{\arabic{enumi}}% Меняем везде перечисления на цифра.цифра
\renewcommand{\labelenumi}{\arabic{enumi}}% Меняем везде перечисления на цифра.цифра
\renewcommand{\theenumii}{.\arabic{enumii}}% Меняем везде перечисления на цифра.цифра
\renewcommand{\labelenumii}{\arabic{enumi}.\arabic{enumii}.}% Меняем везде перечисления на цифра.цифра
\renewcommand{\theenumiii}{.\arabic{enumiii}}% Меняем везде перечисления на цифра.цифра
\renewcommand{\labelenumiii}{\arabic{enumi}.\arabic{enumii}.\arabic{enumiii}.}% Меняем везде перечисления на цифра.цифра

\newcommand{\doublerule}[1][.4pt]{%
	\noindent
	\makebox[0pt][l]{\rule[.7ex]{\linewidth}{#1}}%
	\rule[.3ex]{\linewidth}{#1}}



%\titleformat{\chapter}{}{}{0em}{\bfseries\LARGE\ifnum\value{chapter}>0\ifnum\value{chapter}<3\relax\arabic{chapter}.~\fi\fi}
\titleformat{\chapter}[hang]
{\normalfont\huge\bfseries}{\thechapter.}{20pt}{}



\begin{document}
	\linespread{1.3}
	
	
	\begin{center}
		\textbf{НИТУ <<МИСиС>>}
	\end{center}

	\doublerule
	
	\noindent
	Институт ИТАСУ
	
	\noindent
	Кафедра инженерной кибернетики
	
	\noindent
	Направление подготовки: 01.03.04 Прикладная математика
	
	\noindent
	Квалификация (степень): бакалавр
	
	\noindent
	Группа: \textbf{БПМ-16-2}
	
	\hfill \break
	
	\begin{center}
		\huge\textbf{ОТЧЕТ}\\
		\large\textbf{ПО НАУЧНО-ИССЛЕДОВАТЕЛЬСКОЙ РАБОТЕ} \\
		на тему: <<Система динамического выявления и мониторинга причин клиентских обращений по банковским продуктам>>\\
		\textbf{VIII семестр 2019 - 2020 у.г.}
	\end{center}
	
	\hfill \break
	\hfill \break
	\normalsize{ 
		\begin{tabular}{lccc}
			\textbf{Студент} &  \text{ } & $\underset{\text{подпись}}{\underline{\hspace{4cm}}}$/ & $\underset{\text{Фамилия И.О.}}{\underline{\hspace{5cm}}}$/ \\\\
			\textbf{Руководитель НИР} &  \text{ } & $\underset{\text{подпись}}{\underline{\hspace{4cm}}}$/ & $\underset{\text{Фамилия И.О.}}{\underline{\hspace{5cm}}}$/ \\\\
		\end{tabular}
	}\\\\
	
	\noindent
	\normalsize{ 
		\begin{tabular}{lc}
			\textbf{Оценка:} &  \underline{\hspace{7cm}}\\\\
			\textbf{Дата защиты:} & \underline{\hspace{7cm}}\\\\
		\end{tabular}
	}\\\\
	
	\noindent
	\normalsize{ 
		\begin{tabular}{lcc}
			\textbf{\underline{Утвердил}}: \\\\
			\textbf{Председатель комиссии} &
			$\underset{\text{подпись}}{\underline{\hspace{5cm}}}/$
			&
			$\underset{\text{Фамилия И.О.}}{\underline{\hspace{5cm}}}/$
		\end{tabular}
	}\\\\
	
	\vfill
	\begin{center}
	\textbf{Москва 2020}
	\end{center}
	
	
	\thispagestyle{empty}
	\newpage
	
	
	
	
	
\end{document}